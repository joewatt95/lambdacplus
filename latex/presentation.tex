\documentclass{beamer}
\usepackage[utf8]{inputenc}
\usetheme{Madrid}

\usepackage{amsmath, amssymb, amsthm, thmtools, thm-restate, tikz, graphicx,
  hyperref, cleveref, comment, expl3, xparse, ebproof, enumitem, xcolor,
  stmaryrd, backnaur, verbatim}

\usepackage[titlenumbered, ruled]{algorithm2e}
\usetikzlibrary{shapes.geometric}
\usetikzlibrary{positioning}
\usetikzlibrary{cd}

\usepackage[mathscr]{euscript}

\DeclareMathOperator{\Type}{Type}
\DeclareMathOperator{\FV}{FV}
\DeclareMathOperator{\Pii}{Pi}
\DeclareMathOperator{\Sigmaa}{Sigma}
\DeclareMathOperator{\foralll}{forall}
\DeclareMathOperator{\fun}{fun}
\DeclareMathOperator{\deff}{def}
\DeclareMathOperator{\axiom}{axiom}
\DeclareMathOperator{\checkk}{check}
\DeclareMathOperator{\evall}{eval}
\DeclareMathOperator{\wf}{\mathcal{WF}}
\DeclareMathOperator{\fst}{fst}
\DeclareMathOperator{\snd}{snd}
\DeclareMathOperator{\Kind}{Kind}
\DeclareMathOperator{\inl}{inl}
\DeclareMathOperator{\inr}{inr}
\DeclareMathOperator{\match}{match}

\title[Dependent type theory and Curry Howard]{Dependent type theory and Curry Howard}
\author{Watt Seng Joe \and Abdul Haliq S/O Abdul Latiff}

\begin{document}
\frame{\titlepage}

\begin{frame}
  \begin{block} {What more can we do with types?}
    Formalize mathematics, ie encode theorems and proofs in the computer.
  \end{block}

  \begin{block} {Why?}
    Use computers to help us prove theorems and verify the
    correctness of our programs. \\
    This forms the foundation for interactive and automated theorem proving.
  \end{block}

  \begin{block} {How?}
    Curry Howard says we can encode intuitionistic logic in typed lambda calculi
    \begin{itemize} [label=$\ast$]
      \item Propositional logic $\Longleftrightarrow$ Simple types
      \item Predicate logic $\Longleftrightarrow$ Dependent types
    \end{itemize}
  \end{block}
\end{frame}

\begin{frame}
  \frametitle{Intuitionistic logic}
   \begin{block} {What is it}
      A logic of positive evidence. To say something is true means to
      exhibit evidence in the form of a proof.
    \end{block}

    \begin{block} {What we give up}
      \begin{itemize} [label=$\ast$]
        \item No excluded middle, ie cannot say $\varphi \vee \neg \varphi$ in general, so
          no double negation elimination $\neg \neg \varphi \rightarrow \varphi$

        \item No Axiom of Choice
       \end{itemize}
    \end{block}

     \begin{block} {What we gain}
        Proofs with computational interpretation, ie functions transforming
        evidence of assumptions to that of conclusion.
    \end{block}

\end{frame}

\begin{frame}
  \frametitle{Natural deduction -- Propositional logic}

  \begin{columns}
    \begin{column}{.3\textwidth}
      \textbf{\quad Conjunction}
      \minipage[c][0.5\textheight][s]{\columnwidth}
      \vspace{0.05\textheight}
      \begin{itemize}
      \item
      \[
        \begin{prooftree}
          \hypo{\Gamma \vdash \varphi}
          \hypo{\Gamma \vdash \psi}
          \infer2[$\scriptsize \wedge$-intro]{\Gamma \vdash \varphi \wedge \psi}
        \end{prooftree}
      \]

      \item
      \[
        \begin{prooftree}
          \hypo{\Gamma \vdash \varphi \wedge \psi}
          \infer1[$\wedge$-elim left]{\Gamma \vdash \varphi}
        \end{prooftree}
      \]

      \item
      \[
        \begin{prooftree}
          \hypo{\Gamma \vdash \varphi \wedge \psi}
          \infer1[$\wedge$-elim right]{\Gamma \vdash \psi}
        \end{prooftree}
      \]
      \end{itemize}
      \endminipage
    \end{column}

      \begin{column}{.45\textwidth}
      \textbf{\qquad Implication}
      \minipage[c][0.5\textheight][s]{\columnwidth}
      \vspace{0.05\textheight}
      \begin{itemize}
      \item
      \[
        \begin{prooftree}
          \hypo{\Gamma, \, \varphi \vdash \psi}
          \infer1[$\rightarrow$-intro]{\Gamma \vdash \varphi \rightarrow \psi}
        \end{prooftree}
      \]

      \item
      \[
        \begin{prooftree}
          \hypo{\Gamma \vdash \varphi \rightarrow \psi}
          \hypo{\Gamma \vdash \varphi}
          \infer2[$\rightarrow$-elim]{\Gamma \vdash \psi}
        \end{prooftree}
      \]
      \end{itemize}
      \endminipage
    \end{column}

  \end{columns}
\end{frame}

\begin{frame}
  \frametitle{Natural deduction -- Propositional logic}
  \begin{columns}

     \begin{column}{.6\textwidth}
      \textbf{\qquad Disjunction}
      \minipage[c][0.6\textheight][s]{\columnwidth}
      \vspace{0.05\textheight}
      \begin{itemize}
      \item
      \[
        \begin{prooftree}
          \hypo{\Gamma \vdash \varphi}
          \infer1[$\vee$-intro left]{\Gamma \vdash \varphi \vee \psi}
        \end{prooftree}
      \]

      \item
      \[
        \begin{prooftree}
          \hypo{\Gamma \vdash \psi}
          \infer1[$\vee$-intro right]{\Gamma \vdash \varphi \vee \psi}
        \end{prooftree}
      \]

      \item
      \[
        \begin{prooftree}
          \hypo{\Gamma \vdash \varphi \vee \psi}
          \hypo{\Gamma \vdash \varphi \rightarrow \eta}
          \hypo{\Gamma \vdash \psi \rightarrow \eta}
          \infer3[$\vee$-elim]{\Gamma \vdash \eta}
        \end{prooftree}
      \]
      \end{itemize}
      \endminipage
    \end{column}

    %   \begin{column}{.45\textwidth}
    %   \textbf{\qquad Negation and falsity}
    %   \minipage[c][0.5\textheight][s]{\columnwidth}
    %   \vspace{0.05\textheight}
    %   \begin{itemize}
    %   \item Denoted $\bot$.
    %   \item In intuitionistic logic, we define
    %     $\neg A \equiv A \rightarrow \bot$
    %   \item To say something is false means you have
    %   \item
    %   \[
    %     \begin{prooftree}
    %       \hypo{\bot}
    %       \infer1[$\bot$-elim]{A}
    %     \end{prooftree}
    %   \]
    %   \end{itemize}
    %   \endminipage
    % \end{column}
  \end{columns}
\end{frame}

\begin{frame}[fragile]
  \frametitle{Curry Howard, aka Propositions as Types}

  \begin{block} {The big idea}
    \begin{align*}
      \overbrace{\Gamma}^{\text{Typing env}} \vdash  \quad & \overbrace{1 + 2}^{\text{Term}} \quad : \quad \overbrace{\text{Int}}^{\text{Type}} \\
      \underbrace{\Gamma}_{\text{Assumptions}} \vdash \quad & \underbrace{\text{p}}_{\text{Proof}} \quad : \quad \underbrace{\text{P}}_{\text{Prop}}
    \end{align*}
    \begin{itemize}[label=$\ast$]
      \item Proving a proposition $\Longleftrightarrow$ constructing a term of the
        corresponding type
      \item Checking a proof for correctness $\Longleftrightarrow$ type checking
      \item Simply typed lambda calculus $\Longleftrightarrow$ Intuitionistic propositional logic
      \item Dependent types $\Longleftrightarrow$ First/higher order intuitionistic logic  
    \end{itemize}
  \end{block}
\end{frame}

\begin{frame}
  \frametitle{Typing judgments}

  \begin{columns}
    \begin{column}{.5\textwidth}
      \textbf{\quad Function type} \\ \quad ($\cong$ Implication)
      \minipage[c][0.6\textheight][s]{\columnwidth}
      \vspace{0.05\textheight}
      \begin{itemize}[label=$\ast$]
      \item Intro
        \[
        \begin{prooftree}
          \hypo{\Gamma, {\color{blue} x} : \sigma \vdash {\color{blue} e} : \tau}
          \infer1{\Gamma \vdash {\color{blue} (\lambda \, x, \, e)} : \sigma \rightarrow \tau}
        \end{prooftree}
        \]

      \item Elim
        \[
        \begin{prooftree}
          \hypo{\Gamma \vdash {\color{blue} e_1} : \sigma \rightarrow \tau}
          \hypo{\Gamma \vdash {\color{blue} e_2} : \sigma}
          \infer2{\Gamma \vdash {\color{blue} e_1 \, e_2} : \tau}
        \end{prooftree}
        \]
      \end{itemize}
      \endminipage
    \end{column}

    \begin{column}{.5\textwidth}
      \textbf{\quad Product/Pair type} \\ \quad ($\cong$ conjunction)
      \minipage[c][0.6\textheight][s]{\columnwidth}
      \vspace{0.05\textheight}
      \begin{itemize}[label=$\ast$]
      \item Intro
      \[
        \begin{prooftree}
          \hypo{\Gamma \vdash {\color{blue} a} : \sigma}
          \hypo{\Gamma \vdash {\color{blue} b} : \tau}
          \infer2{\Gamma \vdash {\color{blue} (a, \, b)} : \sigma \wedge
            \tau }
        \end{prooftree}
      \]

      \item Elim left
      \[
        \begin{prooftree}
          \hypo{\Gamma \vdash {\color{blue} (a, \, b)} : \sigma \wedge
            \tau}
          \infer1{\Gamma \vdash {\color{blue} a} : \sigma}
        \end{prooftree}
      \]

      \item Elim right
      \[
        \begin{prooftree}
          \hypo{\Gamma \vdash {\color{blue} (a, \, b)} : \sigma \wedge
            \tau }
          \infer1{\Gamma \vdash {\color{blue} b} : \tau}
        \end{prooftree}
      \]
      \end{itemize}
      \endminipage
    \end{column}
  \end{columns}
\end{frame}

\begin{frame}
  \frametitle{Typing judgments}
  
  \qquad \qquad \quad \textbf{Sum/tagged union type} \, ($\cong$ disjunction)
  \begin{columns}
    \begin{column}{.7\textwidth}
      % \minipage[c][0.6\textheight][s]{\columnwidth}
      \vspace{0.05\textheight}
      \begin{itemize}[label=$\ast$]
      \item Intro left
      \[
        \begin{prooftree}
          \hypo{\Gamma \vdash {\color{blue} a} : \sigma}
          \infer1{\Gamma \vdash {\color{blue} \text{inl } a} : \sigma \vee \tau}
        \end{prooftree}
      \]

      \item Intro right
        \[
        \begin{prooftree}
          \hypo{\Gamma \vdash {\color{blue} b} : \tau}
          \infer1{\Gamma \vdash {\color{blue} \text{inr } b} : \sigma \vee \tau}
        \end{prooftree}
        \]

      \item Elim
        \[
        \begin{prooftree}
          \hypo{\Gamma \vdash {\color{blue} s} : \sigma \vee \tau}
          \hypo{\Gamma, {\color{blue} a} : \sigma \vdash {\color{blue} c} :
            \eta}
          \hypo{\Gamma, {\color{blue} b} : \tau \vdash {\color{blue} d} : \eta}
          \infer3{\Gamma \vdash 
            {\color{blue} (\text{match $s$ with $\vert$ inl $a$ $\Rightarrow$ c $\vert$
                inr $b$ $\Rightarrow$ d end)}} : \eta}
        \end{prooftree}
        \]

        \begin{block} {Key point}
          The pattern for elim must be \alert{complete}, ie you must consider
          both possible cases.
        \end{block}
      \end{itemize}
      % \endminipage
    \end{column}
  \end{columns}
\end{frame}

\begin{frame}
  \frametitle{Towards dependent type theory} 
  \begin{block} {Limitation of simple type theory}
    Cannot express quantified formulae like $\forall x \, \varphi(x)$ since
    types cannot contain variables or other expressions.
  \end{block}

  \begin{block} {Dependent types to the rescue}
    \begin{itemize} [label=$\ast$]
      \item Dependent types offer an elegant solution. 
      \item Key idea is to allow types to \textit{depend on values}.
    \end{itemize}

  \end{block}

\end{frame}

\begin{frame}
  \frametitle{Intuitionistic natural deduction -- Quantifiers}
  \begin{columns}
  \begin{column}{.5\textwidth}
      \textbf{\qquad Universal quantifier}
      \minipage[c][0.5\textheight][s]{\columnwidth}
      \vspace{0.05\textheight}
      \begin{itemize}
      \item
      \[
        \begin{prooftree}
          \hypo{\Gamma \vdash \varphi(x)}
          \infer1[\scriptsize $\forall$-intro]{\Gamma \vdash \forall x \, \varphi(x)}
        \end{prooftree}
      \]
      provided $x$ does not occur free in any hypothesis on which $\varphi$ depends.

      \item
      \[
        \begin{prooftree}
          \hypo{\Gamma \vdash \forall x \, \varphi(x)}
          \infer1[\scriptsize $\forall$-elim]{\Gamma \vdash \varphi(t)}
        \end{prooftree}
      \]
      \end{itemize}
      \endminipage
    \end{column}

    \begin{column}{.5\textwidth}
      \textbf{\qquad Existential quantifier}
      \minipage[c][0.5\textheight][s]{\columnwidth}
      \vspace{0.05\textheight}
      \begin{itemize}
      \item
      \[
        \begin{prooftree}
          \hypo{\Gamma \vdash \varphi(t)}
          \infer1[\scriptsize $\exists$-intro]{\Gamma \vdash \exists x \, \varphi(x)}
        \end{prooftree}
      \]

      \item
      \[
        \begin{prooftree}
          \hypo{\Gamma \vdash \exists x \, \varphi(x)}
          \hypo{\Gamma, \, \varphi(x) \vdash \psi}
          \infer2[\scriptsize $\exists$-elim]{\psi}
        \end{prooftree}
      \]
      provided $x$ is not free in $\psi$ and $\Gamma$.
      \end{itemize}
      \endminipage
    \end{column}
  \end{columns}
\end{frame}

\begin{frame}
  \frametitle{Dependent Pi type (WIP)}

  \begin{columns}
    \begin{column}{.5\textwidth}
      \textbf{\quad Pi type} \\ \quad (generalizes simple function type)
      \minipage[c][0.6\textheight][s]{\columnwidth}
      \vspace{0.05\textheight}
      \begin{itemize}[label=$\ast$]
      \item Intro
        \[
        \begin{prooftree}
          \hypo{\Gamma, {\color{blue} x} : \sigma \vdash {\color{blue} e} : \tau}
          \infer1{\Gamma \vdash {\color{blue} (\lambda \, x, \, e)} : \sigma \rightarrow \tau}
        \end{prooftree}
        \]

      \item Elim
        \[
        \begin{prooftree}
          \hypo{\Gamma \vdash {\color{blue} e_1} : \sigma \rightarrow \tau}
          \hypo{\Gamma \vdash {\color{blue} e_2} : \sigma}
          \infer2{\Gamma \vdash {\color{blue} e_1 \, e_2} : \tau}
        \end{prooftree}
        \]
      \end{itemize}
      \endminipage
    \end{column}

    \begin{column}{.5\textwidth}
      \textbf{\quad Product/Pair type} \\ \quad ($\cong$ conjunction)
      \minipage[c][0.6\textheight][s]{\columnwidth}
      \vspace{0.05\textheight}
      \begin{itemize}[label=$\ast$]
      \item Intro
      \[
        \begin{prooftree}
          \hypo{\Gamma \vdash {\color{blue} a} : \sigma}
          \hypo{\Gamma \vdash {\color{blue} b} : \tau}
          \infer2{\Gamma \vdash {\color{blue} (a, \, b)} : \sigma \wedge
            \tau }
        \end{prooftree}
      \]

      \item Elim left
      \[
        \begin{prooftree}
          \hypo{\Gamma \vdash {\color{blue} (a, \, b)} : \sigma \wedge
            \tau}
          \infer1{\Gamma \vdash {\color{blue} a} : \sigma}
        \end{prooftree}
      \]

      \item Elim right
      \[
        \begin{prooftree}
          \hypo{\Gamma \vdash {\color{blue} (a, \, b)} : \sigma \wedge
            \tau }
          \infer1{\Gamma \vdash {\color{blue} b} : \tau}
        \end{prooftree}
      \]
      \end{itemize}
      \endminipage
    \end{column}
  \end{columns}
\end{frame}

\end{document}